\documentclass{article}
\usepackage[paperwidth = 16.14cm, 
            paperheight = 10cm, 
            textwidth = 16.14cm,
            textheight = 10cm]{geometry}
\usepackage{pgfplotstable}
\usepackage{amsmath,bm}
\usepackage{tikz}

\usetikzlibrary{arrows,
	arrows.meta,
	decorations.pathmorphing,
	calc,%
	decorations.pathmorphing,%
	decorations.markings,
	fadings,%
	shadings,%
	positioning,
	spy,
	shapes,
	shapes.geometric,
	shapes.arrows,
	fit,	plotmarks,}
\usepackage{tikz-dimline}
\tikzset{fontscale/.style = {font=\relsize{#1}}
    }


    \renewcommand{\refname}{\vspace*{-0.3cm}\normalfont\selectfont\footnotesize Referencias} 
 
\begin{document}
\thispagestyle{empty}
\bibliographystyle{ieeetr}
\nocite{*}
    \begin{tikzpicture}[overlay,remember picture,]

\draw[fill=orange, fill opacity=0.3,draw=black](-0.55cm,-0.8cm)--(16cm,-0.8cm)--(16cm,0.29cm)--(-0.55cm,0.29cm)--cycle;


\node[anchor=north west,text width=15.2cm](logo1) at (current page.north west) {\includegraphics[scale=7]
{figures/logo.png}};
\node[anchor=north,text width=7cm, align=center,scale=0.9](title) at (current page.north) {F\'isica 1\\ Ecuaciones de Energ\'ia - Momento\\O. Ruiz-Cigarrillo};



\node[text width=6cm,anchor=north west,yshift=-1.25cm,draw,scale=0.8](n1) at (current page.north west){Ecuaciones de trabajo
\begin{flalign}
     W &= \vec{F}\cdot\Delta \vec{s}\\
     W &= F\Delta s \cos(\theta)
\end{flalign}
};
\node[text width=6cm,anchor=north,yshift=-1mm,draw,scale=0.8](n2) at (n1.south){
Energia cinetica y el principio del trabajo y la energ\'ia.
\begin{flalign}
    K &= \dfrac{1}{2}mv^2\\
    W_{net}&=\Delta K = \dfrac{1}{2} mv_{f}^2-\dfrac{1}{2}mv_{i}^2
\end{flalign}
};

\node[text width=6cm,anchor=north,yshift=-1mm,draw,scale=0.8,align=justify](n3) at (n2.south){
    Una \textbf{fuerza conservativa} es aquella para la cual el trabajo realizado
    por la fuerza al mover un objeto de una posición a otra depende só-
    lo de las dos posiciones, y no de la trayectoria seguida.
};

\node[text width=6cm,anchor=north west,xshift=1mm,draw,scale=0.8,inner sep=4.5mm](n4) at (n1.north east){
Energ\'ia Potencial gravitaciones y energ\'ia potencial elastica. 
\begin{flalign}
    U_{grav} &= mgy\\
    U_{el} &= \dfrac{1}{2}kx^2
\end{flalign}
Cuando sólo actúan fuerzas conservativas, la energía mecánica
total $E$,
\begin{flalign}
    E = K + U
\end{flalign}
Si actuan otro tipos de energ\'ia, la ley de conservaci\'on de energ\'ia
\begin{flalign}
    \Delta K + \Delta U + \Delta (\text{otros})=0
\end{flalign}
La \textbf{potencia} se define como la tasa a la que se efectúa trabajo,
    o la tasa a la cual la energía se transforma de una forma a otra:
\begin{flalign}
    P&=\dfrac{dW}{dt} = \dfrac{dE}{dt}\\
    P&=\vec{F}\cdot\vec{v}
\end{flalign}
};

\node[text width=6cm,minimum height=7cm,anchor=north west,xshift=1mm,draw,scale=0.8,inner sep=3.5mm,](n5) at (n4.north east){
La  \textbf{cantidad de movimiento lineal} o \textbf{momento lineal} $\mathbf{\vec{p}}$, de un objeto se define como el producto de su masa por su velocidad,
\begin{flalign}
    \vec{\mathbf{p}}=m\vec{\mathbf{v}}
\end{flalign}
Cuando la fuerza neta externa sobre un sistema de objetos es
cero, la cantidad de movimiento total permanece constante. Ésta es la
ley de conservación de la cantidad de movimiento.
La ley de la conservación de la cantidad de movimiento es
muy útil al tratar con la clase de eventos conocidos como colisiones.
El \textbf{impulso} de una fuerza sobre un objeto se
define como
\begin{flalign}
    \vec{\mathbf{J}} = \Delta \vec{\mathbf{p}}
\end{flalign}
La cantidad de movimiento total se conserva en cualquier colisión:
\begin{flalign}
    \vec{\mathbf{p}}_{A} +\vec{\mathbf{p}}_{B} = \vec{\mathbf{p}}^{\prime}_{A}+\vec{\mathbf{p}}^{\prime}_{B}
\end{flalign}



};

\node[fill=blue,fill opacity=0.2, anchor=north, text width=6cm, scale=0.8,yshift=-1mm]at (n3.south) 
{   
    \parbox{6cm}{
    \footnotesize
    \vspace*{-0.45cm}
    \bibliography{refs.bib}}
};

\node[anchor=north, text width=6cm, scale=0.8,yshift=-1mm,draw=black,text=black]at (n3.south) 
{   
    \parbox{6cm}{
    \footnotesize
    \vspace*{-0.45cm}
    \bibliography{refs.bib}}
};











\end{tikzpicture}
\end{document}